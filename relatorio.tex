
\documentclass[12pt,a4paper]{report}
\usepackage[utf8]{inputenc}
\usepackage[T1]{fontenc}
\usepackage[brazil]{babel}
\usepackage{graphicx}
\usepackage{array}
\usepackage{hyperref}
\usepackage{longtable}
\usepackage{indentfirst}
\usepackage{setspace}
\onehalfspacing

\begin{document}

% CAPA
\begin{titlepage}
    \centering
    {\Large PROJETO TÉCNICO - PORTFÓLIO PROFISSIONAL \par}
    \vspace{2cm}
    {\Huge \textbf{RELATÓRIO TÉCNICO: BLUETOOTH LOW ENERGY 5.1} \par}
    \vspace{2cm}
    {\Large ANA FLÁVIA MACIEL \par}
    \vfill
    {\large CURITIBA - PR \par}
    {\large 2025 \par}
\end{titlepage}

% FOLHA DE ROSTO / OBJETIVO
\chapter*{Projeto Técnico - Portfólio Profissional}
Relatório técnico desenvolvido como parte de projeto de aprimoramento em redação técnica, com foco na estruturação de conteúdo acessível e tecnicamente embasado para uso em portfólio profissional.

\chapter*{Objetivo}
Relatório técnico desenvolvido com foco na redação e documentação técnica de tecnologias emergentes, tomando como base o Bluetooth Low Energy 5.1. O documento busca apresentar de forma clara, objetiva e tecnicamente embasada a evolução do BLE, seus usos práticos e suas limitações, com linguagem acessível e estrutura de acordo com normas ABNT.

\tableofcontents

\chapter{Introdução}
O Bluetooth é uma tecnologia de comunicação sem fio de curto alcance amplamente empregada em dispositivos móveis, como smartphones, fones de ouvido e vestíveis. Com o crescimento da Internet das Coisas (IoT), surgiram demandas por soluções com maior eficiência energética e funcionalidades ampliadas, como posicionamento em ambientes internos.

Para atender a essas demandas, foi desenvolvido o Bluetooth Low Energy (BLE), ou Bluetooth de Baixo Consumo, introduzido na versão 4.0 da especificação Bluetooth Core. A versão 5.1, lançada em 2019, trouxe melhorias significativas, incluindo novos mecanismos de localização, maior precisão na comunicação e suporte a novas aplicações na indústria, na saúde e na automação. Desde então, versões subsequentes como a 5.2 (2020), 5.3 (2021) e 5.4 (2023) têm trazido aprimoramentos em eficiência, segurança e novas funcionalidades como o LE Audio e suporte a transmissão de dados em multicast com menor latência e maior estabilidade, demonstrando a constante evolução dessa tecnologia.

\chapter{Evolução das versões do Bluetooth}
A tecnologia Bluetooth tradicional, conhecida como BR/EDR (Basic Rate/Enhanced Data Rate), apresentava consumo elevado de energia, o que limitava sua aplicação em dispositivos portáteis. A introdução do BLE na versão 4.0 focou na redução de consumo e simplificação da comunicação.

A versão 5.0 ampliou a taxa de transmissão para até 2 Mbps, o alcance (aproximadamente 240 m em áreas abertas) e a capacidade de broadcast. Já a versão 5.1 adicionou os recursos de localização baseados em direção de sinal (Angle of Arrival -- AoA e Angle of Departure -- AoD), permitindo rastreio mais preciso de dispositivos. A versão 5.2 introduziu o LE Audio, uma tecnologia que substitui o áudio clássico via Bluetooth, promovendo melhor qualidade sonora e menor consumo. A versão 5.3 otimizou a alternância entre canais e o controle de energia, enquanto a 5.4 possibilitou a criação de redes maiores e mais eficientes por meio de melhorias na difusão de dados e no controle de acesso.

\chapter{Tabela Comparativa}
\begin{longtable}{|c|c|p{6.5cm}|p{6.5cm}|}
\hline
\textbf{Versão} & \textbf{Ano} & \textbf{Destaques} & \textbf{Limitações} \\ \hline
4.0 & 2010 & Introdução do BLE, baixo consumo & Baixa taxa de transmissão \\ \hline
5.0 & 2016 & Maior alcance, velocidade e capacidade & Sem suporte à localização precisa \\ \hline
5.1 & 2019 & Localização por AoA/AoD, direção de sinal & Requer hardware compatível \\ \hline
5.2 & 2020 & LE Audio; isocrônico para áudio simultâneo & Início da transição, adoção parcial do mercado \\ \hline
5.3 & 2021 & Otimização de consumo e troca de canais & Pequenas melhorias incrementais \\ \hline
5.4 & 2023 & Suporte avançado a redes mesh; melhorias no controle de acesso & Complexidade maior para integração \\ \hline
\end{longtable}

\chapter{Casos de Uso}
\begin{itemize}
    \item \textbf{Localizadores pessoais}: etiquetas inteligentes (como AirTags) utilizam BLE 5.1 para localizar objetos com alta precisão em ambientes fechados.
    \item \textbf{Navegação indoor}: sistemas de localização interna em hospitais, aeroportos e centros comerciais.
    \item \textbf{Monitoramento industrial}: rastreamento de equipamentos e ativos em tempo real em linhas de produção.
    \item \textbf{Saúde e bem-estar}: pulseiras fitness e monitores cardíacos que operam por dias com uma única carga.
    \item \textbf{Controle de acesso}: portas inteligentes que se abrem com base na aproximação de dispositivos autorizados.
    \item \textbf{Automação residencial}: sensores de movimento e temperatura integrados com hubs domésticos inteligentes.
    \item \textbf{Transmissão de áudio via LE Audio}: auriculares e aparelhos auditivos com qualidade superior e menor consumo.
\end{itemize}

\chapter{Posicionamento com AoA e AoD}
As técnicas Angle of Arrival (AoA) e Angle of Departure (AoD) permitem que dispositivos compatíveis com BLE 5.1 determinem a direção de onde o sinal está vindo ou para onde está sendo enviado, com base em múltiplas antenas e medições de fase.

\begin{itemize}
    \item \textbf{AoA}: o receptor (por exemplo, um smartphone) mede a direção do sinal utilizando duas ou mais antenas para calcular o ângulo de chegada.
    \item \textbf{AoD}: o transmissor (como um beacon) envia sinais com variações de fase entre antenas, permitindo que o receptor calcule o ângulo de origem.
\end{itemize}

Conforme ilustrado na Figura~\ref{fig:aoa_aod}, essas técnicas permitem que os dispositivos realizem estimativas de direção do sinal com base na análise de fase entre múltiplas antenas, tanto no transmissor quanto no receptor.

Essas técnicas permitem estimativas de localização com margem de erro inferior a 1 metro, viabilizando aplicações em rastreamento de ativos, navegação indoor e segurança em ambientes controlados.

\begin{figure}[h!]
    \centering
    % Substitua o arquivo abaixo por uma imagem na mesma pasta do .tex (por exemplo, figura_aoa_aod.png)
    \includegraphics[width=0.75\textwidth]{figura_aoa_aod.png}
    \caption{Funcionamento do posicionamento com AoA e AoD. Fonte: EBYTE (2021).}
    \label{fig:aoa_aod}
\end{figure}

\chapter{Limitações}
Apesar das inovações, o BLE 5.1 apresenta algumas limitações:
\begin{itemize}
    \item Exige hardware específico com múltiplas antenas para funcionamento completo de AoA/AoD.
    \item Interferência em ambientes congestionados (Wi-Fi, micro-ondas) pode afetar a precisão.
    \item Alcance inferior ao de tecnologias como Wi-Fi, dependendo do ambiente.
    \item Posicionamento com limitações em locais com muitas barreiras físicas ou reflexões de sinal.
\end{itemize}

\chapter{Conclusão}
O Bluetooth Low Energy 5.1 representa um avanço significativo na tecnologia de comunicação sem fio de curto alcance, unindo eficiência energética e recursos de localização precisos. Sua aplicação se estende por diversos setores, contribuindo para o desenvolvimento de soluções inteligentes e integradas.

As versões posteriores, como 5.2, 5.3 e 5.4, reforçam esse avanço ao introduzir novos paradigmas de áudio, gerenciamento energético e escalabilidade em redes. A constante evolução do Bluetooth Low Energy mostra que esta tecnologia continuará sendo uma base sólida para inovações na conectividade de dispositivos e na expansão da Internet das Coisas.

\chapter{Referências}
\begin{itemize}
    \item BLUETOOTH SIG. \textit{Bluetooth Low Energy Primer}. Disponível em: \url{https://www.bluetooth.com/bluetooth-le-primer}. Acesso em: 23 jun. 2025.
    \item EBYTE. \textit{Um minuto para entender os princípios de medição de distância com AOD e AOA}. 2021. Disponível em: \url{https://www.ebyte.com/news/1461.html}. Acesso em: 24 jul. 2025.
    \item WOOLLEY, M. \textit{Bluetooth Core 5.1 Feature Overview}. Bluetooth SIG, 2019. Disponível em: \url{https://www.bluetooth.com/blog/bluetooth-core-specification-5-1/}. Acesso em: 23 jun. 2025.
\end{itemize}

\end{document}
